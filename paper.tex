\documentclass[runningheads]{llncs}





\usepackage{color}

\usepackage{graphicx}
\usepackage{tikz}
\usepackage{calc}

\usepackage{amsmath, amssymb}
\DeclareMathOperator*{\argmin}{arg\,min}

\usepackage{wasysym}

\usetikzlibrary{automata}




\title{On finiteness of structures produced deterministically by oritatami co-transcriptional folding
\thanks{This work is supported in part by KAKENHI Grant-in-Aid for Challenging Research (Exploratory) No.~18K19779 and JST Program to Disseminate Tenure Tracking System No.~6F36 granted to S.~S.}
}
\author{
Szil\'{a}rd Zsolt Fazekas\inst{1} \and
Shinnosuke Seki\inst{2}\thanks{Corresponding author}}
\institute{
Akita University, 
Graduate School of Engineering Science, 
1-1 Tegate Gakuen-machi, Akita, 0108502, Japan \\
\email{szilard.fazekas@ie.akita-u.ac.jp}
\and
The University of Electro-Communications, 
Graduate School of Informatics and Engineering, 
1-5-1 Chofugaoka, Chofu, Tokyo, 1828585, Japan \\
\email{s.seki@uec.ac.jp}
}

\begin{document}

\maketitle

\begin{abstract}

\end{abstract}

\section{Introduction}


\section{Preliminaries}
Let $\Sigma$ be a set of types of abstract molecules, or \textit{beads}. 
By $\Sigma^*$ and $\Sigma^\omega$, we denote the set of finite sequences of beads and that of one-way infinite sequences of beads, respectively. 
A bead of type $a \in \Sigma$ is called an $a$-bead. 
Let $w = b_1 b_2 \cdots b_n \in \Sigma^*$ be a sequence of length $n$ for some integer $n$ and bead types $b_1, \ldots, b_n \in \Sigma$. 
The \textit{length} of $w$ is denoted by $|w|$, that is, $|w| = n$. 
For two indices $i, j$ with $1 \le i \le j \le n$, we let $w[i..j]$ refer to the subsequence $b_i b_{i+1} \cdots b_{j-1}b_j$; if $i = j$, then $w[i..i]$ is simplified as $w[i]$. 
For $k \ge 1$, $w[1..k]$ is called a \textit{prefix} of $w$. 

Oritatami systems folds their transcript, which is a sequence of beads, over the triangular grid graph $\mathbb{T} = (V, E)$ cotranscriptionally. 
A directed path $P = p_1 p_2 \cdots p_n$ in $\mathbb{T}$ is a sequence of \textit{pairwise-distinct} points $p_1, p_2, \ldots, p_n \in V$ such that $\{p_i, p_{i+1}\} \in E$ for all $1 \le i < n$. 


\begin{proposition}\label{prop:check_validity}
	For any rule set $R$, arity $\alpha$ and conformation $C = (P,w,H)$ it is possible to check whether $C$ is $R$-valid and whether $C\in \mathcal{C}_{\leq \alpha}$ in time $\mathcal{O}(|H|\cdot|w|\cdot|R|)$.
\end{proposition}
\begin{proof}
	To check whether $C$ is $R$-valid:
	\begin{enumerate}
		\item FOR each $(i,j)\in H$:
		\item \hspace{1cm} IF $(w[i],w[j])\notin R$ THEN answer NO and HALT
		\item answer YES and HALT
	\end{enumerate}	
	Checking the condition in 2. can be done in $\mathcal{O}(|w|\cdot|R|)$ time for any reasonable representation of $w$ and $R$, hence the whole process takes $\mathcal{O}(|H|\cdot |w|\cdot|R|)$ time.	
	To check the arity constraint $C\in \mathcal{C}_{\leq \alpha}$: 
	\begin{enumerate}
		\item FOR each $i\in \{1,\dots,|w|\}$:
		\item \hspace{1cm} IF $\mathrm{degree}(i)=|\{j | (i,j)\in H \}|>\alpha$ THEN answer NO and HALT
		\item answer YES and HALT
	\end{enumerate}	
	Checking the condition in 2. can be done in $\mathcal{O}(|H|)$ time for any reasonable representation of $H$, hence the whole process takes $\mathcal{O}(|w|\cdot|H|)$ time.
\end{proof}
\begin{theorem}\label{thm:OS_to_2dTM}
	There is an algorithm that simulates any oritatami system $\Xi = (\Sigma, R, \delta, \alpha, \sigma, w)$ in time $2^{\mathcal{O}(\delta)}\cdot |R|\cdot|w|$. 
\end{theorem}
\begin{proof}
	Take any step in the computation, up to which some $i \ge 0$ first beads of $w$ have been stabilized, with the last bead at a point $p$. 
	The number of all possible elongations of the current conformation by the next $\delta$-beads is $(6 \times 5^{\delta-1}) \times ((2^4)^{\delta-1} \times 2^5) \in 2^{O(\delta)}$. 
	By Proposition~\ref{prop:check_validity}, we can check for each of these elongations whether its arity is at most $\alpha$ or not and whether it is $R$-valid or not in time $\mathcal{O}((2^4)^{\delta-1}\cdot2^5\cdot \delta\cdot|R|)=2^{\mathcal{O}(\delta)}\cdot|R|$.  Therefore, the total running time is $2^{\mathcal{O}(\delta)}\cdot |R|\cdot|w|$.
\end{proof}

\begin{corollary}
	For fixed $\delta$ and $\alpha$, the class of problems solvable by oritatami systems $(\Sigma, R, \delta, \alpha, \sigma, w)$ is included in $\mathrm{DTIME}(n^3)$.
\end{corollary}
\begin{proof}
	The claim follows from Theorem~\ref{thm:OS_to_2dTM} and the fact that $|R|$ is implicitly bounded by $|w|^2$.
\end{proof}

%\begin{corollary}\label{polytranscript}
%	Let $k$ be a non-negative number. Let $\Sigma$, $H$, $\delta$, $\alpha$ be fixed and the transcript length bounded polynomially by the seed length, i.e., $|w|\in O(|\sigma|^k)$. Then, the language of seeds $\sigma$ accepted by an OS $(\Sigma, R, \delta, \alpha, \sigma, w)$ is in $\mathrm{DTIME}(|\sigma|^{2k})$.
%\end{corollary}

%If the transcript length is polynomially bounded by the seed length, than the accepted language is in $\mathrm{P}$. 

Because of the time hierarchy theorems, we know that $\mathrm{P}\subsetneq \mathrm{EXP}$, so we can conclude that OS which cannot deterministically fold transcripts of length exponential in the length of the seed are not computationally universal.

\section{Case of $\delta = 1$}


\section{Case of $\delta \geq 2$}


\subsection{$\alpha = 1$, unary}

Let the point where the first transcript bead was fixed be $p$ and let $n=|\mathrm{seed}|+1$. We will argue about the situation when the first bead is stabilized outside $\hexagon_p^n$ (a hexagon of radius $n$). Let this be the $i$th bead of the transcript. Without loss of generality, we can translate the origin $(0,0)$ to the coordinates of bead $i-1$ (which is still in $\hexagon_p^n$), and we can assume that the bead outside the hexagon is fixed at $(1,1)$ (see Fig.~\ref{fig:hexagonOut}).
\begin{figure}
	\centering
	\includegraphics[width=0.3\linewidth]{./Fig/hexagonOut}
	\caption{$N_p^n$ and the position $(1,1)$ of the first bead fixed outside of it.}
	\label{fig:hexagonOut}
\end{figure}

In the elongation that places bead $i$ at $(1,1)$ there are two possibilities:
\begin{itemize}
	\item $i$ forms a bond with a bead at $(1,0)$.
	\item  $i$ does not bond to anything and $i+1$ is at $(2,1)$ bonding with a bead at $(2,0)$. If there is no bead at $(1,0)$, then placing $i$ at $(1,0)$ instead of $(1,1)$ results in the same number of bonds, leading to nondeterminism. Therefore, there has to be a bead at $(1,0)$ and it is inert, otherwise it would bond to $i$. This is analogous to case 1. below.%as in Fig.~\ref{fig:hexagonOut1}.
	\begin{figure}
		\centering
		\includegraphics[width=0.2\linewidth]{./Fig/hexagonOut1}
		%\caption{}
		\label{fig:hexagonOut1}
	\end{figure}
	
\end{itemize}
 
%The only position with which $(1,1)$ can form a bond is $(1,0)$. This means that there is a bead at $(1,0)$, which bonds to bead $i$, otherwise there are other conformations in which beads $i$ and $i+1$ add one bond to the conformation, making the behavior nondeterministic.

The next bead, $i+1$, can be fixed at $(2,1)$ or at $(0,1)$ as all other possibilities result in nondeterministic behavior immediately, so we have two cases.

\begin{enumerate}
	\item bead $i+1$ is fixed at $(2,1)$ and can bond with a bead at $(2,0)$. Now consider bead $i+2$. For $i+1$ to be fixed at $(2,1)$, $i+2$ needs to form a bond somewhere, otherwise $i+2$ could go to $(2,1)$ forming the bond with the bead at $(2,0)$ and there would be two conformations with the maximal $1$ bond. The only possibility is that there is a bead at $(3,0)$ and $i+2$ can bond with it when placed at $(3,1)$. We can apply the same argument inductively: there is some $m\geq 0$ such that grid points $(\ell,0)$ are occupied by active beads, for all $\ell\in \{2,\dots,2+m\}$, and there is no bead at $(3+m,0)$. Such an $m$ exists, and it is not greater than $n$. Then, bead $i+\ell$ is fixed at $(\ell+1,1)$ and bonds with $(\ell+1,0)$. However, bead $i+2+m$ cannot be fixed anywhere, because $i+2+m$ and $i+3+m$ can only add one bond to the conformation, and that is possible either with $i+2+m \rightarrow (2+m,1)$, $i+3+m \rightarrow (3+m,1)$ or with $i+2+m \rightarrow (2+m,2)$, $i+3+m \rightarrow (2+m,1)$. 
	\begin{figure}
		\centering
		\includegraphics[width=0.6\linewidth]{./Fig/hexagonOut2}
		\caption{}
		\label{fig:hexagonout2}
	\end{figure}
	
	\item bead $i+1$ is fixed at $(0,1)$. This is only possible if
	\begin{enumerate}
		\item there is an inactive bead at $(-1,0)$ and an active one at $(-2,0)$. This case is symmetrical to (1).
		\item there is no bead at $(-1,0)$, bead $i+1$ can bond with bead $i-1$ at $(0,0)$ and the bead $i+2$ can be placed at $(-1,0)$ where it can bond with $(-2,0)$, $(-2,-1)$ or $(-1,-1)$. This leads to nondeterminism, because bead $i$ at $(-1,0)$ and bead $i+1$ at $(0,1)$ has two bonds, just as the original conformation.
		\item there is a bead at $(-1,0)$ and bead $i+1$ can bond with that or with bead $i-1$ at $(0,0)$. However, this means that placing bead $i$ at $(0,1)$ at bead $i+1$ at $(1,1)$ creates the same number of hydrogen bonds, thus resulting in bead $i$ not being placed deterministically.
		
	\end{enumerate}
\end{enumerate}




\begin{figure}
	\centering
	%\includegraphics[width=0.3\linewidth]{./hexagonOut1}
	%\hspace{10mm} %
	\includegraphics[width=0.6\linewidth]{./Fig/hexagonOut3}
	
	\caption{}
	\label{fig:hexagonOut2}
\end{figure}

  
\end{document}
